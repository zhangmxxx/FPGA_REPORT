\documentclass{report}
\usepackage{graphicx}
\usepackage{ctex}
\begin{document}

\begin{titlepage}

  \begin{center}
  
  
  % Upper part of the page
  
  \textsc{\LARGE 实验一:选择器}\\[1.5cm]
  
  {邮箱:211240005@smail.nju.edu.cn}\\[1.5cm]
  
  
  % Title
  
  
  % Author and supervisor
  \begin{minipage}{0.4\textwidth}
  \begin{flushleft} \large
  \emph{姓名:} 张明轩\\
  \emph{学号:} 211240005
  \end{flushleft}
  \end{minipage}
  \begin{minipage}{0.4\textwidth}
  \begin{flushright} \large
  \emph{班级:}一班 \\
  
  \end{flushright}
  \end{minipage}
  
  \vfill
  
  % Bottom of the page
  {\large 2022.9.19}
  
  \end{center}
  
  \end{titlepage}
\newpage
\section*{一、实验目的}
通过学习几种常用的多路选择器的设计方法,学习掌握Verilog语言的always语句块用法,以及分支语句的使用
。通过自行设计二位四选一选择器,熟悉通过使用vivado进行电路设计的基本流程。
\section*{二、实验原理}
多路选择器的输入端为多路输入数据和一位或多位选择控制端,输出端口为一路输出数据。
其功能为,通过选择控制端,选择多路输入数据中的某一路,并将其输出到数据输出端口。
电路设计中,某些多功能器件的实现(例如ALU)并不是通过控制信号指定执行某一种计算,
而是先计算出所有结果,再通过多路选择器选择输出结果。
\section*{三、实验环境、器材}
实验环境:Vivado2022.1;
实验器材:Nexys A7-100T
\section*{四、程序代码/流程图}
\section*{五、实验步骤}
\subsection*{设计}
\subsection*{编译}
\subsection*{仿真}
\subsection*{写入}
\subsection*{硬件验证}
\section*{六、测试方法}
使用仿真测试和实验平台测试。其中,仿真测试枚举了所有控制信号的值,并枚举该控制信号对应的输入端口的值,
观察对于不同的控制信号,被选择端口与输出端口是否同步改变。\\
在实验平台上,通过控制开关完成和仿真测试一样的测试内容,同时,更改非选择输入端口的值,观察是否对
实验结果产生影响。
\section*{七、实验结果}
对于不同的控制信号,被选择端口与输出端口同步改变,且非选择端口的值对该结果不产生影响。
\section*{八、遇到的问题和解决方法}
在always语句块的敏感事件列表中,只包含了控制端口Y,导致当Y不变而被选择端口值改变时,输出端口未同步改变。
将敏感事件列表用*代替后解决。
\section*{九、实验得到的启示}
verilog支持用数字电路的方式描述电路,也可以通过always语句块等,通过高级语言来进行描述。前者更接近实现电路,
后者更接近“功能描述”。从不同的层面去思考,能够加深对该电路的理解。
\section*{一、意见和建议}
可以提供一些与实验无关的verilog编程内容,方便更加快速的掌握
\end{document}
